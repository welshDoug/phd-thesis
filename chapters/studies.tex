\chapter{Case Studies}  %1000
\label{ch:studies}
Having reviewed the state of the art of spatio-temporal analysis through the three components identified at the beginning of this thesis, the next step is the application of the conclusions from that review. This will demonstrate how the application of `interesting' questions, answered by appropriate methodology, using spatio-temporal tools is an improvement to the status quo of archaeological spatio-temporal analysis, which is the ultimate objective of this thesis. This demonstration will be via case studies on two data sets, one focusing on a particular site, Hambledon Hill, at a local scale. The other taking the much larger perspective of a continental scale database for the early Neolithic, the EUROEVOL data set \citep{Manning:2016fk}. The case studies will show at different scales how space and time may be re-combined for a more holistic analysis of the temporal record. The use of case studies is essential as there are no example studies clearly showing such an improved spatio-temporal methodology. It is not sufficient to talk in general terms of improved spatio-temporal analysis, as such discussion is purely speculative. It is also necessary to practice the ideas that one has preached, to move from speculation to a concrete example, demonstrating how such ideas can be put to practical application. 

Both of the chosen data sets have previously been the subject of intensive temporal analysis, the case studies here will seek to re-combine the spatial and temporal evidence for the areas under study. Such an approach will make the identification of the benefits of combined spatio-temporal analysis more obvious and offer enhancements for our understanding of the sites and processes under examination.

\section{Hambledon Hill}
The first case study will focus on Hambledon Hill, it will evaluate the temporal aspect of the archaeological record for the site and the analysis of that record, to determine the validity of the interpretation offered in the original report, \citet{Mercer:2008fk}. This will require a review of the temporal evidence and of how the evidence supports the existing interpretation. The case study will then undertake a combined spatio-temporal analysis of the data set, focusing on how the examination of a combined record can provide both a critical review and an enhanced interpretation compared to the original, based only on temporal data. This case study will be focused around theories and method; firstly those employed in the original investigation, specifically the bayesian analysis of dates, secondly of the spatial limitations and opportunities available with such a method, and finally an examination using combined spatio-temporal approaches. The scale of this study is that of the local level and just like the original report, at that level, the focus is on detailed, contextualised examination of the available data. At this scale it is possible and profitable to examine the evidence for particular events and consider how these events might have been enacted or experienced by individuals in the past. This study will make a valuable contribution to the practice of such local scale, subject centred archaeological study if it is successful in demonstrating how combined spatio-temporal analysis can aid with the detailed examination of the data.

\section{Spread of the European Neolithic}
The second case study will examine the EUROEVOL data set, \citep{Manning:2016fk} a continental scale database of radiocarbon dates. It will critically review the existing methods used to analyse the data set, to determine their suitability for answering the questions asked of them. The study will then undertake an analysis of the data set itself, the potential limitations and concerns of the make up of the data set. It will also examine analytical approaches used when working with such large data sets, considering in the abstract how analytical methods can support interpretation of evidence. Following this, the study will perform combined spatio-temporal analysis of the data set demonstrating the benefits of alternative methods for understanding the make up of the data set and for drawing conclusions from it. As with the first case study, the analysis of theory and method are a core element of the study. In this case the focus is more around the applicability of methods to the data set, examining those that have already been used and evaluating the potential of other methods. The scale of this study is broad, analysing change at a continental level, as such the interpretations skip detail, are generalising and reductionist. At this scale the data set is simply too large for a data point-by-data point evaluation of the database, meaningful and carefully chosen abstractions are required to enable conclusions to be drawn. The use of such abstractions means that the interpretations are likely to consider people in the past in abstract or aggregate forms, unable to access the individual experience and instead examining processes and the routines that govern them. This study will be successful if it also demonstrates how combined spatio-temporal analysis can develop our understanding of the data set and enhance interpretations of it.

The two studies provide an insight at both ends of the spectrum of scale, clearly there are many differences in the questions that are asked of such varying scales of data and in the methods that are applicable. There are differences in terms of the level of detail, from the very detailed and contextualised to the abstract and generalised. There are also fundamental differences in the schools of thought of the original analysis, with the SPD approach firmly sitting as a neo-processualist and the detail focused, subject centred approach of \citet{Whittle:2011kl} could be described as fitting into a broad post-modernist doctrine. Despite the broad divides in the case studies and these philosophical traditions, the two studies are related by focus on combined spatio-temporal analysis and the central position given to the review of theory and method in both studies. While differences will be required when working with such different scales of data, both in method and theory, such differences need not be of the magnitude that splits archaeology between neo-processualists and subject centred archaeologists. Following the studies, the review will examine the differences between the studies, to present the conclusions on a unified archaeological spatio-temporal analysis. 

