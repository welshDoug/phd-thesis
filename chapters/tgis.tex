\chapter{Spatio-Temporal Software}  %7770
\label{ch:tgis}
\section{Approaches to Archaeological Temporal GIS}
To make the most of archaeological data it is necessary to ask questions that consider both the spatial and temporal dimensions. In order to answer such questions, appropriate analytical techniques are required, such as custom methods for extending temporal techniques to perform combined spatio-temporal analysis. The platform for performing such analysis will therefore need to work with combined spatio-temporal data, clearly an appropriate choice for such a platform is temporal GIS. There has been to date, as \citet{Green:2008fk} put it, only a niche interest from archaeologists into temporal geographic information systems, with perhaps as few as two generic pieces of software being created. This has not stopped archaeologists asking spatio-temporal questions and devising methods for analysis that incorporate temporal information, recent examples including \citet{Frachetti:2006fk}, \citet{Palmer:2006uq} and \citet{Whittle:2008fk}.

The first consideration of the topic was \citet{Castleford:1992fk} who provides an overview of the state of play in the early '90s, at the time other disciplines were seriously considering temporal GIS (TGIS), especially from a theoretical perspective, such as \citet{Langran:1992uq}. Castleford brought this together with temporal archaeological studies and more general philosophy of time to propose some of the choices that would have to be made for an archaeological TGIS. For example, deciding whether temporal models should be points or intervals, or whether temporal databases would be better as relation or object oriented. While he does not propose the details for a specific system, his examination of some of the issues has laid the groundwork for subsequent researchers to come along and attempt to solve those problems. At the time he suggested archaeologists should become involved with the technical people creating TGIS software, however while TGIS has become commonplace in some fields, there are few re-usable pieces of software for archaeology. 

Castleford's research was undertaken at a time when there was considerable interest in Temporal-GIS from a wide range of subjects. \citet{Langran:1992uq} provides a perspective from the geographers approach, starting with an overview of the subject, an examination of what she calls ``cartographic time''  \cite[27]{Langran:1992uq} i.e. time as it is represented in a TGIS. This starts with a brief analysis of time from a philosophical perspective, however Langran quickly pushes this to one side in order to progress with the more pragmatic question of time in cartography. Unfortunately for archaeologists, Langran's cartographic time is precise and the temporal model has a fine granularity. This means that it is of limited use, as archaeological dates are often probabilistic and an additional source of temporal information is from the topological relationship between features. 

What is more useful for archaeologists is the more general information about the capabilities and requirements of a TGIS, for example \citet{Langran:1992uq} lists six major TGIS functions as: Inventory, Analysis, Updates, Quality Control, Scheduling and Display. This list is similar to that of \citet{Wheatley:2002ly}, with inventory being analogous to their spatial database. The main difference is the quality control and scheduling functions, which do not appear on the \citet{Wheatley:2002ly} list. 

\citet{Langran:1992uq} also lists the key technical requirements of a TGIS as:
\begin{itemize}
\item Conceptual model of spatial change
\item Treatment of aspatial attributes
\item Data-processing logistics
\item Spatio-temporal data access method
\item Efficient algorithms to operate on spatiotemporal data
\end{itemize}
The efficiency of algorithms is a significant part of her work, however it was written over 20 years ago, computing power and storage facilities have developed significantly since then. While these considerations are still very important, it is perhaps less crucial now than it was then. The other four technical requirements however are just as important.

In addition, Langran discusses some of the kinds of queries and operations that might be expected from a TGIS. This includes answering questions such as: where and when did change occur? What types of change occurred? What is the rate of change? What is the periodicity of change? It also includes assessing temporal data in order to determine whether temporal patterns exist, if so what trends are apparent and to try to determine what processes underlie the change.

These general requirements have formed a key part in many TGIS studies in archaeology since. However, it took several years before any follow up to Castleford's initial call to arms was made, the first being \citet{Daly:1999fk} who reviewed the progress on the study of time in archaeology since Castleford. They found that while there had been much work on the study of temporality in archaeology, there had been little progress on the implementation of TGIS. They discuss some of the issues faced by archaeologists in this area, and noted that there had been plenty of progress on the implementation of TGIS in other fields, in particular geography. They review several potential approaches to modelling time taken from geography, focusing on how relevant such approaches might be to archaeological applications. Most approaches suffer from the problem that while they may be able to show change, they do not necessarily help in analysing that change, for example in the snapshot approach where multiple raster's are stored as time slices, there is no actual information stored about the change, all it allows us to do is measure change by comparing values from different images \citep{Daly:1999fk}. The authors go on to suggest three potential methods for modelling and representing time which they think are suitable for archaeological applications: object-oriented database, animation and a z-axis. They note there are very few examples of object-oriented systems and the technical threshold is quite high, they are wary of animations due to the inherent lack of any analysis capability, which being key to any GIS system leaves animation more as a form of potential output, but not a temporal model. Their favoured approach is the z-axis, as used by many space-time geographers, a theoretical description of such a system for archaeological use is presented in \citet{lock2002analysing}, discussed below. \citet{Daly:1999fk} also offers up some of the potential uses for a TGIS, such as the ability to study how people in the past experienced and interacted with time. They suggest the time-space life worlds of time geographic research may be applicable to cultures and societies. This would make such techniques more practical for archaeological data, with the temporal rhythms of societies manifested in the spatial configuration of their activities \citep{Daly:1999fk}. While the article provides a valuable review of the literature as of 1998, it does not add any tangible technology, it does however provide a potential direction for the future of TGIS in archaeology.

\citet{Lock:1999fk} provides an interesting re-usable approach to studying time through GIS, while it  is not a packaged system, it does offer a method for including many of the features archaeologists might expect to find in such a system. The temporal model for this approach relies on pottery typology and quantitative data of pottery find distributions (i.e. from field walking) where different time periods are represented by different types of pottery. By splitting the finds for an area based on their type, they analyse how pottery deposition is different for each type, therefore how it changed over time and use this to infer a change in activity levels on a site over time. In its basic form this is done by subtracting the two quantities represented as raster images. This study takes that analysis a step further, using classification to aid visualisation and by suggesting two further methods, one for examining the change relatively, and the other for doing qualitative analysis. These extensions to the method can be used to account for the significance of different types of pottery, and the classification enables the use of contingency tables for further analysis. Clearly, this is a limited model, crucially reliant on pottery typologies, but its simplicity does enable the researcher looking at large areas to start to visualise how that area might have changed over time, identify the most profitable parts to investigate for a particular period, or to try to understand change between periods. By focusing on continuity or change the temporal model fits well into a raster based system, it can provide output that is clear and intuitive, however the low level of resolution has the potential for creating authoritative looking outputs that mask the true complexity of the underlying model, of the actions and events that led to each act of deposition. Using their suggested techniques for relative and qualitative comparisons would enable the archaeologist to control this to a degree, but not alleviate it entirely. 

The difficulties of visualising temporal data in GIS were carefully considered in \citet{lock2002analysing}. The approach they recommended was designed to break the constraints that they saw on the representation of change through time in 2/2.5D views. They argue that existing GIS systems lead to temporal data being simplified, as they cannot natively support a temporal dimension. Time is then often reduced to a specified span, or a period code for each feature, stored either as an attribute, or using temporally organised layers. This oversimplification is not only a reduction in the quality of data, but may also influence the study of change, for example with a focus on comparing characteristics of clearly defined periods that in reality are not so clearly defined \citep[4]{lock2002analysing}. With respect to visualisation, they are critical of using colour coding of sites to represent time, due to the potential for creating a complex mess of colours and for the problems of sites that span periods. The fundamental problem they see with these categorical techniques is that time is continuous, and therefore must be analysed as such.

The Temporal-GIS system they propose to account for these problems has never been created, so its efficacy in practice is unknown. It is fundamentally based on probabilities, enabling it to deal with the uncertainty of archaeological dating. The probabilities are called p(use) values, they represent the probability that a site was in use at a particular point in time. This allows archaeologists to easily model sites, which have a period of time in which they were definitely in use and the areas either side of this where they may have been in use. The p(use) values for a particular site are computed based on the p(use) values for each piece of evidence from that site, with the recommended approach being to take the greatest of all the values. The result being a p(use) timeline for the site, which is used for a 3D spatio-temporal representation. The p(use) values are defined by the archaeologist, enabling them to incorporate a range of knowledge about each piece of evidence. At first glance this is an attractive system, however, to approach a continuous z-axis would require an archaeologist to place a p(use) value on every piece of evidence for each site along the z-axis at a high enough resolution that meaningful analysis could be performed. The problem being that our knowledge of use may be highly discrete, we may have a very good idea of when something was in use, and when it was not in use, but very little detail on the transition. The solution is a continuous p(use) function, but there is no clear way of providing this for a variety of different evidence types, such as radiocarbon, pottery typology and site morphology. In addition, with such a breadth of potential evidence being one of the stated benefits of this method, to focus on one type of dating evidence alone would severely impact its usefulness. There is also a further problem to placing all the p(use) values at the archaeologists' discretion, that of lack of rigour. Without a system or framework for computing such values, it is more difficult to argue about how they have been reached and very easy to argue about the values themselves. 

One of the fundamental goals from \citet{lock2002analysing} was a means for visualising temporal data. Even if there are problems with the underlying temporal model, the visual aspects to their approach have their own merits. As an intuitive display of temporal data a 3D approach with continuous z-axis is very appealing, as they note it has the potential for using topology to explore spatio-temporal patterns and for performing analysis of temporal data \citep[5]{lock2002analysing}. There are a couple of concerns with this approach, firstly, in a crowded landscape viewing 3D representations on a screen could prove very tricky, in particular it will be difficult see sites that are both near and ones that are further away from the origin at the same time. This could potentially result in sites in a part of the landscape at one point in time appearing to be spatially much nearer to sites in another part, further along the y axis, at a different point in time. There is also the issue of probability, this means that the temporal relationships between sites will likely also have a probabilistic nature, so too will any analysis along the z-axis. The result of these issues is that, at best, it is going to be very easy to get undecipherable screens, and depending on the allocation of p(use) values users may end up with ``garbage in - pretty maps out'' \citep[1]{lock2002analysing}.

Following this, \citet{Ceccarelli:2003fk} attempted to construct a spatio-temporal GIS to model landscape change, focusing on the area around a lake in Tuscany. Their solution involved a time database combined with spatial modelling, using animation to represent time as a third dimension for output. However, the changes to the shoreline are not visualised in the animation as a continuous change, but as four distinct  versions of the coastline, shown in sequence. Within their database, this makes sense, as they treated time as a categorical value, the different visualisations of the landscape presumably corresponding to one or more categories. It is curious that they have then created an animation, with a continuous time variable, and as \citet{Green:2008fk} notes, it is a bit deceptive. It would appear that by using a long list of software they have created a visually appealing animation displaying multiple layers from a GIS sequentially, where the layers show the same landscape at different points in time, and contain key archaeological and historical information overlaid. As with other systems a large focus was on creating a means of representing change through time to the public in an easily consumable fashion, rather than performing any kind of temporal analysis. This project was also a very bespoke solution, using lots of software, including: AutoCAD, Access, ArcGIS, Grass, PostgreSQL, RDBMS, 3D Studio Max and QuickTime. It was by no means re-usable temporal GIS software.

\section{Temporal GIS Implementations}
\citet{Johnson:1999cr} was the first attempt at creating a distinct archaeological temporal geographic information system, the TimeMap project. The TimeMap software created for the project has been very successful; it was used in the Museum of Sydney for a kiosk application, the Electronic Cultural Atlas Initiative, and as a map interface for educational software MacquarieNet \citep{Johnson:2002kx}. However, it hasn't been used in many archaeological projects, despite the lack of TGIS for archaeological uses being one of the key drivers behind creating the software. Green neatly summarises the situation ``TimeMap's greatest successes have been in areas outside of core TGIS issues for archaeologists and whilst using non-archaeological data'' \citep[101]{Green:2008fk}. The reason for this is down to the initial objectives. In fact, there was no specific target application in mind \citep{Johnson:1999cr} and this has allowed the project to be pulled off course. \citet{Johnson:1999cr} explores the subject of archaeological TGIS and defines some distinct requirements for such a system, including the ability to deal with uncertainty in temporal data, he also defines the temporal model to be used - the snapshot-transition model. The first projects the TimeMap software was used for are the mapping of Asian empires and Historic Sydney \citep{Johnson:1999cr}. While neither is a strictly archaeological project, there is potential for the inclusion of archaeological data. TimeMap seems to have changed direction by \citet{Johnson:2002kx}, as there is much less focus on Archaeology and the issues around archaeological TGIS, but much more focus on education and interactive mapping as an educational tool. This is exemplified with the change in focus of the projects now involved with TimeMap, to museum and education applications. This shift was accompanied by an increasing focus on the use of the web, the networking capabilities became a much more important part of TimeMap, but are hardly essential for an archaeological TGIS. 

In terms of its use as an archaeological TGIS TimeMap suffers from a number of problems, its snapshot-transition model does not record the topological relationship between features \citep{Johnson:1999cr}. This is a key form of temporal relationship in archaeological data. Another important format of temporal data in archaeology is the probabilistic, such as radiocarbon dates, while \citet{Johnson:1999cr} discusses the importance of fuzzy dates and a mechanism for coping with them, there is no indication of applying the technique to probabilistic dates. Without support for topological or probabilistic dates, TimeMap's use within archaeology is quite limited. Presumably this is why the datasets that have been used are all of a historical nature, they can provide exact dates down to specific years, which is also much easier to represent. \citet{Johnson:2002kx} presents the space-time cube as the ultimate representation of spatio-temporal data. The space-time cube depicts accurate and exact information, however to represent a probabilistic date in this way could easily imbue it with a false accuracy or exactness. With hindsight there are many issues with the project, but TimeMap was the first bespoke ``archaeological'' TGIS, to create it meant finding answers, or at least pragmatic solutions to some of the problems for which earlier authors had only hypothesised solutions. It has laid the foundation for subsequent work in archaeological Temporal GIS, primarily \citet{Green:2008fk}. 

\citet{Green:2008fk} has moved the discipline a significant step closer to an archaeological temporal GIS. It can be seen as a proof of concept for the technique of providing bespoke archaeological temporal facilities within ArcGIS and demonstrates the potential for TGIS studies in archaeology. The thesis starts with a comprehensive review across the literature, from philosophy of time, through archaeological theory, the study of time, to digital archaeology and the history of the study of temporal GIS in archaeology. The only notable exception being the use of GIS in archaeology. The software created by Green is a plug-in for ArcGIS, which he describes as a ``fuzzy temporal GIS'' \citep[242]{Green:2008fk}. 

The primary function of the software is to compute the probability that a date, or set of dates fall within a defined time span, as such it only deals with absolute dates, it does not consider relative dates. These probabilities can then be drawn in ArcGIS using the spatial co-ordinates of the dated object, with the probability value represented on screen using a mechanism, such as colour, to indicate the likelihood it falls into the time span. In practice Green used several time spans for each of his case studies, where a time span represents a period that the archaeologist wishes to analyse (e.g. Late Neolithic) this results in a series of outputs for each site, presenting a set of snapshots visualising the probabilities that the mapped dates occur in each time span, figure~\ref{fig:green1} is an example of such a snapshot. There is no indication whether the dates are more likely to have occurred during a particular part of the time span, so in many ways the result is not too dissimilar from a snapshot model, there is also no information about the transitions between each phase, or time span. Unlike the snapshot model, output can be generated for any time span without the need for interpolation. With large datasets these outputs quickly become messy, with coloured points overlaying each other. For his first case study, the intra-site example, points were overlaid on a base map containing features from multiple periods, utilising the same base map for each run of the system, even though each run represented a consecutive period. The resulting output retained eligibility only due to the sparseness of the temporal data set. There is also no relationship in the system between the dates, represented as points, and the features. The lack of a clear operational model to Green's TGIS is a big omission, without encapsulating change in the system its effectiveness is very limited. As a conceptual model of change is one of Langran's key technical requirements of a TGIS, this does raise the question of whether it is indeed a TGIS at all.

Data is input by augmenting the shapefile containing point data, so that each entry also contains a minimum and a maximum date as attributes, apart from OxCal data, which is contained in a separate table, linked to the shapefile within ArcGIS. This is a convenient method of getting temporal data into ArcGIS, but Green's labelling it as the temporal model for the system is unconvincing. The temporal model that is encapsulated in this system clearly draws on the B-series notion of time as it deals with dates BC/AD or BP. The minimum and maximum dates specified are used to define a period during which the date occurred, rather than a beginning and end date for the object or event. It is also not clear what dates in the data model represent; when radiocarbon dates are used, they represent a specific event causing the sample to cease its exchange of Carbon, whereas dates from other sources may have different meanings. Surely, an important element of the temporal model is what the dates actually represent. Another issue with the data model is that it only deals with point dates, so it cannot consider the life of an object, it is in effect a one dimensional temporality, one that does not store change, or life spans. 

The TGIS of \citet{Green:2008fk} had one primary form of analysis, the calculation of probabilities that the temporal entities within the system fell into a user specified date range. In his case studies this was then used as the basis for colour coding the plotted data. It also provides several secondary analysis methods, which are not applicable to the main data type, radiocarbon dates, as they involve summing probabilities.

These analytical methods are solely focused on the temporal data, with the primary function, computing probability, creating secendary data that can lead into further spatio-temporal analysis. The use of interpolation is problematic, Green notes many problems himself \citep[189]{Green:2008fk} especially for sparse datasets, such as his radiocarbon dataset. Fundamentally though, it is essential to consider the underling assumptions the interpolation is based on, that locations closer to certain dates are more likely to have dates closer to that date than ones further away. The results of interpolation like this are attempting to show the spatial distribution of the probability that dates occurred during a specific time interval, i.e. the likely focus of dated material during that time interval. However with a sparse data set, likely to have been heavily influenced by post depositional processes, considered across a relatively small scale area, the underlying assumptions of the use interpolation is, perhaps questionable and the generated probabilities could just as easily reflect the likelihood of finding material as identifying areas of focus of past activities. Clearly spatio-temporal interpolation is a tempting analytical method, as Green uses it twice, however it must be treated with caution. Green's second use of interpolation, \citet[206]{Green:2008fk} based on pottery data is performed with a clear question in mind and provides a much more robust set of results. It is also over a much larger area and is looking for a more general trend. By using trend surface analysis Green demonstrated broad trends in the distribution of different pottery that had similarities to earlier, conventional interpolations of pottery data. With a large data set, over the area of a county this can demonstrate clear areas of focus during particular periods. However there is still a very basic problem, that if a large part of the area has dates that have a broad potential range of possible dates, there will be minimal distinction between the different date ranges. 

Despite its limitations, \citet{Green:2008fk} is a watershed moment, from research that is mostly about how we might do T-GIS to actually writing bespoke software. In addition, he was able to demonstrate its usefulness in two very different contexts. Crucially he was able to combine the statistical, technical, almost processual approach to data, with the kind of post processual theory and analysis that leads to an evidence based humanising of people in the past.

\section{Next steps for Archaeological Temporal GIS}
There is clearly a gulf between the theory of archaeological T-GIS and the implementation. Moreover, implementations that are less driven by the archaeology, such as TimeMap, ultimately end up being much less useful for archaeologists. This is perhaps mostly due to the disconnect between T-GIS theory and the spatio-temporal data available to archaeologists, as a large part of the founding theory has been written by geographers. The other disconnect between theory and application, as exemplified by \citet{lock2002analysing} is that while it is important to consider temporal models and ideal ways of representing probabilistic spatio-temporal information, it is an entirely different thing to implement those models and render the output. To bridge this gulf, pragmatism is clearly essential, only heeding those aspects of T-GIS theory that are applicable to archaeological applications and data. 

This pragmatism has so far been best demonstrated by \citet{Green:2008fk} in his use of the ArcGIS platform, while this comes with being limited to the ArcGIS SDK, in practice that is not particularly restrictive and it provides access to a mature GIS environment. Despite nearly 25 years since the first interest from the archaeological sector, the use of T-GIS in archaeology is still in its infancy, a fully featured platform is a long way off. At the moment archaeological T-GIS are single use applications, providing a subset of available analytical methods to answer specific questions, although the list of true spatio-temporal methods is itself very limited at present. Over time, more methods will be added to the list, so that we will one day be in a position to build an environment that can handle archaeological spatio-temporal data and perform a wide range of useful analysis.

Having reviewed the available T-GIS created within archaeology, it is clear there are non that are capable of performing the kinds of analysis required to enhance the field of spatio-temporal analysis, specifically by incorporating spatial data into temporal analysis. Instead it will be necessary to create bespoke approaches to answering individual questions, with such approaches incorporation a subset of T-GIS capabilities that are necessary to perform a particular analysis. At this stage too much focus on re-usable systems is not particularly helpful, instead, a focus now on spatio-temporal questions and spatio-temporal methods will be crucial in defining the T-GIS of the future. 

\section{Essential Temporal GIS Components}
There are several components of T-GIS systems that will be necessary for performing any combined spatio-temporal analysis, however they can be tailored to the specific methods and therefore be considerably less complex than those required for general T-GIS applications.

\subsection{Spatio-temporal Model}
The spatial model used by typical GIS systems is essentially one of cartesian space. In order to model the world more accurately, layers will have various projections onto this space so that they can represent what is on the ground. Ultimately the GIS deals with co-ordinates, often 2D, sometimes in 3D. This is very much along the lines of the absolute concept, where space is a container of objects \citep{James-Conolly:2006qf} or perhaps in some ways analogous to the B-series notion of time. Such an objective model of the world is important in that it allows the combination of data from disparate sources in such a way that it is easy to view everything related to a specific area, even if that information comes from a variety of data sources. It is also the view of the world that most forms of cartography use, meaning that maps and surveys fit straightforwardly into GIS.

Just as the A-series is the subjective counterpart in terms of time, the relative concept of space makes it an attribute of objects, events or people \citep{James-Conolly:2006qf}. From this perspective it is possible to examine how people understand the space around them, ask questions of being-in-the-world and of the experiential nature of space. This is an area that modern GIS studies in archaeology have started to tackle. Crucially it has been possible to ask questions about experience and perception from within a GIS where the underlying spatial model is absolute.

There is no reason why this same approach cannot be taken with time, \citet{Green:2008fk} makes use of absolute dates as the basis for his temporal model. One of the objectives of the T-GIS then becomes providing a framework for analysing time as experienced. The same approach was taken by \citet{Johnson:1999cr} in TimeMap and also the continuous z-axis of \citet{lock2002analysing}. In fact there are few examples that do not use absolute dates, as these are one of the most readily available to archaeologists.

By taking this approach to time it is important not to assume that it is theoretically neutral, in the past GIS was considered as such, however it has been demonstrated that this is in fact not the case \citep{Wheatley:1993qf}. So far, the theoretical nature of time has revolved around splitting time along the lines of the A/B series theory of time as adapted by \citet{Gell:1992fk}, in particular with his adaptation of Husserl's model of Time-consciousness. On this basis, to attempt to understand how people in the past understood time we must consider the A-series. \citet{Green:2008fk} drew a connection between Husserls ideas of reproduction, retention and protection with the broad themes from \citet{Bradley:2002fk} of people in the past looking into their distant past, their immediate past, looking towards the future and people in the past encountering ancient remains. 

There are many other theoretical approaches to time, even though the A/B series sits neatly alongside the predominant spatial model it is worth considering other approaches as they may have valuable insights for the temporal model.

The Annales school approach to time could be applicable to a TGIS approach, with the individual dates being stored in the temporal database representing short term events. The challenge is to knit these into longer term discourses and explore the longer term process behind the short term events. This in some ways parallels archaeological interpretation, the objects found and dates associated with them are of events taking place in the short term. However the understanding of past societies, process and cultures will take place over a longer time frame. Such an approach could favour the representation of changes to society as happening over the long term, when in fact this may not have been the case. Instead non-linear systems could provide a model for more sudden, dramatic change. Potentially this could fit well with the tighter date ranges yielded by bayesian modelling of radiocarbon dates.  Coming from a different theoretical standpoint, time-geography may have some difficulties with the types of data archaeologists have available, for example constructing a time budget is much more challenging as there is little insight into the lives of particular individuals. However the more general ideals of taking social ideas and representing them in physicalist terms could hold potential, perhaps as a method of visualising archaeological assumptions. Such a temporal model would probably have to be quite advanced, but if achievable might yield interesting results when combined with biographical approaches. Another theoretical model of time is time economics and opportunity costs. In the post-processual world of modern archaeology such an approach could be divisive, but could yield interesting results about the way people in the past made use of their time. With models and simulations of opportunity costs it would fit well into GIS analysis. 

Clearly there is a wide variety of potential influences from which to draw for the spatio-temporal model. The specific theoretical foundations will depend upon the analytical method used and the ultimate questions that are being addressed. The broadest theoretical approach would appear to be the A/B-series as advocated by \citet{Gell:1992fk}, which will work well with existing spatial models. The Annales school and non-linear dynamics both incorporate the idea of different scales of time, and focus on change. The representation of change is referred to in \citet{Castleford:1992fk} as the operational model of the TGIS.

\subsection{Operational Model}
\citet{Langran:1992uq} suggests several conceptual models of time: space-time cube, sequential snapshots, base state with amendments, space-time composite. Of these approaches, Langran favoured the space-time composite, while \citet{Johnson:1999cr} chose sequential snapshots, and the z-axis approach of \citet{lock2002analysing} is a derivation of the space-time cube, that allows for probabilistic dates. 

The space-time cube is perhaps the most visually appealing, with ``the trajectory of a two-dimensional object through time create[ing] a worm-like pattern in this phase space'' \cite[37]{Langran:1992uq} however as \citet{lock2002analysing} noted there are problems with the uncertain nature of archaeological data. Langran also notes technical and conceptual problems with this approach, especially with large data sets and the representation of change. This is of critical concern to an archaeological system and there are multiple options, for example, does it interpolate between known data or instead represent change in a stepwise fashion \citep{Langran:1992uq}? \citet{lock2002analysing} suggest a way of displaying uncertainty, and their p-use values effectively provide a mechanism for manual interpolation of change.

The sequent snapshot approach is analogous to a set of maps with each time slice being a different map in the sequence \citep{Langran:1992uq}. For Langran, the problem with this model is that the snapshot represent states, but they do not represent the events that capture the change. In order to determine change the snapshots must be thoroughly compared \citep{Langran:1992uq}. The snapshot-transition model used by Johnson is similar to this, but as well as the snapshots, it also stores information about how to transition between snapshots. Crucially interpolation is required to visualise a state that is not the exact one recorded by a snapshot. 

Base state with amendments provides a much more efficient means of storing temporal data, with change explicitly encapsulated in the amendments \citep{Langran:1992uq}. It is possible to move individual objects forwards and backwards in time, and to analyse the mutations between the amendments. There are no archaeological implementations of this model.

The space-time composite is in many respects an extension of the base state with amendments model except that rather than image overlays, the amendments become new objects in the system, where each object has its own attribute history, represented by an ordered list of records \citep{Langran:1992uq}. Langran details how such a system could be implemented, yet so far no archaeological examples have been constructed, despite the detailed instructions. This is perhaps due to the inexact and incomplete nature of archaeological data, we may have information about a change, but it may be only partial information with a probabilistic date, where as the space-time composite model relies on exactness.

According to Johnson the snapshot-transition model replicates our knowledge about the past \citep{Johnson:1999cr}, however \citet{Green:2008fk} disagrees, in its raw form he states that archaeological temporal data is point based and topology based. However Greens simplification disregards the fact that, ultimately a dated sample represents an event, something that happened to an object (e.g. burning) and that object has a physical manifestation, one that forms a part of the topology of the site. The event represents a change, one which may also be represented in the chronology of the site. In fact the topology of the site is a record of the changes that have occurred, this view of change is in many ways similar to that of \citet{Ingold:1993zr} where the topology of features in the landscape is the physical manifestation of changes made to that landscape over time. 

However, taking a typical method of storing chronology, the Harris-Matrix, would only store the chronological events, it is an a-spatial model of change. Clearly this is only half the answer, and a TGIS must also be capable of storing spatial change as well. In many ways there are parallels to Langrans space-time composite, where space and time are stored separately in two databases \citep{Langran:1992uq}. Archaeologists most frequently represent spatial change with successive plans and diagrams, so it also makes sense to follow Langran and store the chronology separately to the plan as the spatial data is best stored using conventional GIS formats. The two structures should be linked so that it is possible to model the changing spatial extent of a feature throughout its chronology. While there may be certain scenarios that would favour one form of operational model over another, the pragmatic approach is clearly to adopt a form of the space-time composite, with linked data stored in parallel, although not necessarily as a base state plus amendments model.

So far there has been a recurring theme of data, without a consideration of what form that data might take. Having considered spatio-temporal and operational models, it is important to consider what data will be available and how that can be stored in such a way as to realise these models.

\subsection{Data Model}
So far the loosely recommended approach is to use absolute dates for the spatio-temporal model, with a combination of chronology and plans comprising the operational model. There are a range of absolute dating methods available to archaeologists, the most frequently used method is undoubtably radiocarbon dating; the added complexity of radiocarbon dates is that they are in fact a probability. Related to this is their requirement for calibration. \citet{Green:2008fk} uses the OxCal method for storing calibrated dates, which splits the date up into a series of slices, storing the probability that the actual date falls into each slice. 

The chronology of a site could potentially be represented using one of several schemes, with the most common probably being the Harris-Matrix, from \citet{Harris:1989vn}. It has been shown that the Harris-Matrix can be efficiently and effectively stored and analysed digitally using a Directed Acyclic Graph (DAG) data structure \citep{Ryan:1988ly,Herzog:1991ve}. Another potential format would be the OxCal model used as input to the bayesian modelling algorithms, this would aid re-use of existing data and provides a clear connection between temporal and operational models. While both of these models encapsulate change, they embody a very different kind of change. Whereas the Harris-Matrix diagram is a representation of the physical stratigraphy, the OxCal model is a hypothetical model of the sequence of events. Effectively the OxCal model can be thought of as being at a greater level of abstraction than the Harris-Matrix, which is more at the coal face of the chronology. This is due to the interpretation of the site's chronology by the model builder, having translated the changes in features, in stratigraphy and phasing of the site into the abstract form of the OxCal model. Finally, let us consider the relationship between the chronological model and spatial features. For a Harris-Matrix the spatial data can be used to show changes in the physical objects represented in the model. The OxCal model does not represent physical objects, it represented events. It is these events which are the vehicle of physical changes, so in order to show change it would need plans of the actual changes to be linked to the events in the model.

\subsection{Generating Output}
The representation of temporal data and display of results from analysis of temporal data is a subject of its own right. While humans have an inherent visual understanding of space, the same is not necessarily true for time. Time can be represented as a clock or as a line, but there are inherent problems with such methods requiring high precision and being rigidly linked to modern clock time, or at least a B-series style temporal model. 

This has not put off \citet{lock2002analysing}, who suggest the use of 3-D structures such as voxels. They recommend their p(use) method to overcome the problem of uncertainty in archaeological data and the probabilistic nature of dates. This has been discussed previously, from a visualisation perspective the problem is with displaying a spatial entity that may or may not be present at a particular point in time, and representing that uncertainty. In the case of p(use) the absolute value of which should be represented in some way by the display, so elements with a high p(use) are obviously distinct to those with a low value. There is then the question of how to display changes in physical form, the uncertainty of when the change took place and also displaying changes to use. Part of the problem here is the assumption that a spatio-temporal GIS must surely have a method for visualising both time and space together, in some abstract, intuitive form. This is an unhelpful assumption, and just as GIS view space from a different perspective to the one we would experience it (with the omnipresent, God's eye view) there is no reason the same cannot be true of time. Such an abstraction from the human perception of space is often seen as a criticism, removing us from the world, however it is a convenient and often helpful way of viewing the world when performing spatial analysis.

In order to bring the viewer back into the world, virtual reality and animation are often employed to represent spatial data from the perspective of someone actually present in the landscape. In a similar way, several attempts have been made to use animation to represent time in an intuitive way, although spatially these have been from the top down perspective, they are similar in that they represent time as it is experienced, by its passage. This has been the favoured approach to representing time in TimeMap \citep{Johnson:2002kx} and also in \citet{Ceccarelli:2003fk}. While it has it admirers, animation suffers from a number of key problems, firstly it is not interactive, a user cannot perform selections or manipulate data, once in an animation, it is shut off from any further manipulation. Secondly it does not help with the issue of uncertainty in the archaeological data, while transitions can be created by fading features, this implies a certainty of a gradual change.

\citet{Green:2008fk} took a different approach, rather than creating a spatio-temporal display, he only visualises the probabilities that the TGIS produces. The temporal aspect is reflected in the parameters of the analysis and are therefore a static part of the resulting output. In other words the augmented shapefile that is the output to the system is a product of the date range used to work out the probability values, he then uses colour to visualise the probabilities created. \citet{Johnson:1999cr} refers to this as symbolism, one of four techniques he suggests for squeezing an extra dimension out of a computer screen, the others being time slices, arrows and difference maps, see figure~\ref{fig:time-maps}. In effect Green is mapping the variation of probability over space, using colour as an extra dimension (or half dimension) as the main axis are taken up by the spatial co-ordinates.

There are clearly a wide range of potential techniques for visualising spatio-temporal data. The applicability of a 3-D view to archaeological data is unconvincing, primarily due to the uncertainty of probabilistic dates, instead it is more important to focus effort on the analysis and providing a suitable means of representing those results. Animation would certainly help with tackling being-in-the-world issues and can provide an immersive experience, but it is not an analytical tool. One of the aims of outputs is clearly to display the available data, but as \citet{tufte1983visual} notes they also serve as ``instruments to help people reason about quantitative information'' \citep[91]{tufte1983visual}. However, he is also clear on the limitations of the passage of time as a form of causal explanation, \citep[37]{tufte1983visual} suggesting that any diagrams displaying time should only do so as necessary to demonstrate the causes of change to the dataset over time. A good example of this is Minard's graphic of Napolean's Russian campaign (from \citealp[41]{tufte1983visual}), where time is not plotted as an axis, but varies as a function of space. Such a technique is clearly applicable to an advancing army, as it cannot be in two places at once, however that particular technique is less suitable for an archaeological site, which does not move (at least not necessarily in the same way). 

This therefore raises the question of how spatio-temporal data should be plotted, there are three points made by Tufte, which it is essential that any means of displaying archaeological spatio-temporal information follows:
\begin{enumerate}
\item ``graphical elegance is often found in simplicity of design and complexity of data'' \citep[177]{tufte1983visual}
\item ``statistical graphics are instruments to help people reason about quantitative information'' \citep[91]{tufte1983visual}
\item ``have a narrative quality, a story to tell about the data''  \citep[177]{tufte1983visual}
\end{enumerate} 

The final point in \citet[191]{tufte1983visual} is that the task of the graphical designer is the ``revelation of the complex''. Clearly these archaeological datasets are very complex, making the need for techniques to plotting such data in an accessible fashion even more important. However approaches such as the space-time cube and the p(use) voxels are unlikely to meet this principle, due to the approach they take of representing the complexity of the data in its entirety. Archaeological temporal data must not only display changes in space and time, it must also represent the probabilities associated with those values, most specifically with time. A fundamental problem here is that the graphic has been closely associated with the tool, inseparably bound up in the few attempts at an archaeological T-GIS so far. Where as, a corollary of point three above, is that graphics should clearly be more closely bound to the underlying questions that are being asked of the data. The method used to answer those questions will clearly have an impact, but analysis of the graphic should make it clear how a researched came to their answers of the questions posed. 

This chapter has analysed the theory and implementations of temporal GIS for archaeological application and found all available packages to be lacking in the fundamental area of spatio-temporal analytical capabilities. In order to perform such methods it will therefore be necessary to create bespoke implementations, for the specific analytical method being used. This will necessarily involve some specific temporal GIS functionality, such as a combined spatio-temporal data model, an operation model to represent change and fundamentally will contain a spatio-temporal model that is the embodiment of a way of thinking about and working with both space and time. This chapter has highlighted some important considerations for the graphical display of spatio-temporal information, that they should not be treated as a simple means of displaying data, but are a fundamental part of the process of answering archaeological questions. 

